\begin{center}
{\bf \Huge Abstract}

\end{center}

Remote observation of the Earth has always been a point of interest to humans. Over the years, the methods used for this purpose have evolved until, at present, the analysis of multispectral images constitutes a very active line of research, in particular to carry out fire monitoring and follow-up, natural disasters, chemical spills or other types of environmental pollution.

Satellite imagery in a world where machine learning and data processing has advanced so far opens up the possibility of building real-time processing models witch's recognizes areas where a natural disaster has occurred, and being able to act accordingly.

This Final Degree Project carry out the optimization of a machine learning model used to detect natural disasters with the Intel OpenVINO toolkit. In addition, the application is deployed in a Google cloud environment, with the objective of support thousands requests per minute.


%\vspace{1cm}

\vspace{0.8cm}
\begin{center}

{\bf \Large Keywords}

\end{center}


Multispectral images, OpenVINO, TensorFlow, Docker, Google Cloud.
\vspace{0.5cm}

\mbox{}
