\cleardoublepage

\chapter{Conclusions}
\label{apenddixE}

In this project we explored various alternatives for the development of low cost whiteboards that also add functionality to the existing market. We have seen how, with the output of 3D sensors on the market, considerably extend the capabilities of these devices at affordable prices.\\ 
Using one of these devices (\textit{Leap Motion}), we have developed a system that can allow both interaction with the desktop and emulate a blackboard that allows easy storage and sharing all the notes.\\ 
In the field of \textit{traditional} whiteboard (in example, 2D positioning the mouse pointer), we have studied alternatives including a very low cost homemade solution from ultrasound and infrared technology. In the remainder of this section we will study the cost of this solution and discuss its possible implementation in real environments.

% * Coste TOTAL.
% * Necesidad de tener portatil, proyector...
% 		- Posibilidad de portar a raspi (sin leap).
% * Soluciones actuales mucho más caras.
\section{About the manufacturing cost}
\label{makereferenceE.1}

In chapter~\ref{ch:chapter2} we saw the current situation of digital whiteboards and its prices.
Our main objective since then has been trying to reduce the price as much as possible at a minimum lose of quality.\\ 
\\ 
The material cost in detail in one of our sensors, as seen 
in figure~\ref{figureE.1} is approximately 23\euro. The breakdown 
is estimated as follows:
\\
\begin{itemize}
\item STM32VlDiscovery board - 10\euro
\item 40Hz ultrasonic sensor - 2\euro
\item Infrared sensor - 2\euro
\item Integrated circuits (LM393 y TL081) - 3\euro
\item Analogic devices - 1\euro
\item DC-DC converter - 2\euro
\item 3.3V TTL USB to serial wire - 3\euro
\end{itemize}
\begin{figure}[h]
	\centering
	
	\includegraphics[height=3in]{figures/protosensor.jpg}
	\caption{Implemented prototyped sensor.}
	\label{figureE.1}
\end{figure}
\begin{figure}[h]
	\centering
	
	\includegraphics[height=3in]{figures/protolapiz.jpg}
	\caption{Implemented prototyped pencil.}
	\label{figureE.2}
\end{figure}
The most expensive item in this list is the STM32VlDiscovery board, which we do not actually completely need. Just the processor is fine, which can be bought separately for less than 5\euro.\\
\\
Our solution to emulate a mouse with these sensors, as seen in section~\ref{makereference4.2.1}, needs two of these sensors plus a transmitter (as seen in figure~\ref{figure5.2}), whose cost does not exceed 4\euro~retail manufactured.\\
\\
If we add up all the costs, we estimate a cost of about 50\euro~in developing the prototype used to position the cursor. While we must 
note that under a wholesale marketing the cost of manufacturing would be reduced very 
notably, we can not disregard that the current prototype is to demonstrate the feasibility of the system, and 
commercial implementation could lead to increased costs in order to improve current faults.\\
\\
If we add to this the cost of a \textit {Leap Motion} (about 90\euro), we obtain 
gesture recognition with a solution just for 140\euro, compared to prices of the solutions we saw in chapter~\ref{ch:chapter2}.\\
\\
We have to keep in mind that these calculations are not assuming a very important cost, which is the cost of a projector and a computer for the
use of hardware and software implemented. This clearly exceeds the cost 
cost of the sensors (many other solutions have this added cost), but can be neglected, since a projection installation is very common in any classroom. Still, in order to minimize the cost of the computer is 
very likely to carry the ultrasonic solution without \textit{Leap Motion} to a \textit{Raspbery Pi} system, for example.

% * Viabilidad de posicionamieto con ultrasonidos e infrarrojos.
% * Problemas implementación Lapiz digital.
%		- Posibles soluciones.
\section{About viability of our positioning solution}
\label{makereferenceE.2}
Not far from a real solution, our prototype allows us to detect implementation problems about detecting position by 
ultrasonic and infrared. Below are a list of problems 
explained and possible solutions:

\begin{itemize}
\item \textbf{Ultrasonic transmitter being directional:}\\
Unfortunately we have found ultrasonic trasmitters to be 
highly directional. This means that if the transmitter does not point roughly 
directly to the sensors, is relatively easy to lose the signal. 
One possible solution is to use a more omnidirectional transmitter or even use 
more transmitters pointing in different directions.

\item \textbf{Loss of precision with noise or attenuation:}\\
It may occasionally happen that the first oscillation of the ultrasonic receptor is lost or attenuated delaying the ultrasonic wavefront 25 $\mu$s (cycle time of ultrasound). It is easy to see that this delay causes a small error in the measurement of approximately 10 mm according to the formulas views in section~\ref{makereference4.2.2}. For 
solving this little error, we can either improve the amplification stage in the receiver, or implement a singal treatment in application level.

\item \textbf{Signal jamming by malicious student:}\\
It is relatively easy to jam the infrared signal from the transmitter with another 
transmitter (such as a remote control) correctly directed towards the sensors bar, causing the entire system to be unusable. This problem is hardly treatable because of the very low signal / noise ratio in channel.

\end{itemize}

% * Gestos y naturalidad.
\section{About using digital whiteboards in classrooms}
\label{makereferenceE.3}

During development of the project we had meetings with teachers for obtaining a list of needs to be met by a digital whiteboard. In 
this section we report some of them and how they could be covered.\\
\\
In \textit{Colegio Público Asunción Rincón} teachers have a whiteboard enabled classroom. This has a proximity projector and a resistive surface. The cost of this installation is around 1500\euro. 
Teachers use these whiteboards for interactive activities with children, like games of relating figures with names.\\
For this type of activity very high accuracy is not required, so it could easily be implemented with a 
3D sensor as the \textit{Kinect} system.\\
\\
In \textit{Colegio Gredos San Diego} teachers are interested in being able to interact with the 
blackboard by writing on it, in order to save what was wrote. 
This center has shown interest in our approach as it would allow them to position the cursor accurately enough.