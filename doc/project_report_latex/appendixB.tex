\cleardoublepage
\newpage
%\thispagestyle{empty}
\mbox{}

\chapter{Conclusions and future work}
\label{Appendix:Conclusion}

\section{Conclusions}

In this end-of-degree project, the design and implementation of the ATDCA-GS algorithm has been carried out, using Gram Schmidt orthogonalization in order to optimizing and improving the performance of complex operations such as the calculation of the inverse of a matrix. The programming languages VHDL and OpenCL have been used and the results of their execution in FPGA have been evaluated to subsequently make a performance comparison between both alternatives.

As part of the design, an adaptation of the algorithm to the usual flow of a specific hardware design has been carried out, minimizing as much as possible the amount of resources to be used and parallelizing the operations carried out during the execution of the algorithm.

To make a comparison in terms of the performance in time of the two implementations, it has been compared the acceleration of one with respect to the other making use of real and synthetic images. In addition, it has been verified that, except for the implementation in OpenCL for large images, the processing in both alternatives does not exceed the time limit (maximum) and therefore the real-time analysis can be performed, fulfilling one of the main objectives of this proyect.

The performance tests in terms of resources used in each implementation have revealed that the percentage of resources used increases linearly with the number of bands. It also revealed that for a large number of them (256), the resource with the highest use hardly reaches 86\% of use in VHDL and 48\% in OpenCL, so it is concluded that the performance is adequate.

\section{Lines of future work}

In the first place, it would be convenient to improve the implementation optimizations in OpenCL so that it allows a real-time analysis as well as the other alternatives. In addition, since the tendency of the size of the images is to continue growing more and more, the future work option that seems more evident is to be able to process other real images of an even larger size.

A possible future work path for this work would be to develop the algorithm by converting the floating-point arithmetic to whole arithmetic. In this way a better performance would be obtained since the calculations would be even simpler and, therefore, the number of necessary resources would decrease while increasing the clock frequency.

Another possible way of continuation could be the modification of the test platform to use the PCIe 3x8 bus. In this way penalties due to I / O would be reduced.

Finally, another way would be to choose whether the implementation kernels in OpenCL can follow a parallel programming model at task level, so that the task refers to the execution of a kernel with a work-group that contains a work-item and, thus, the compiler tries to accelerate the only work-item to get a better performance.