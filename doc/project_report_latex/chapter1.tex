\cleardoublepage


\chapter{Introducción}
\label{ch:chapter1}


\section{Motivación}

Una imagen hiperespectral es una imagen de gran resolución espectral que se obtiene a través de sensores capaces de obtener cientos o incluso miles de imágenes sobre el mismo área terrestre
pero correspondientes a diferentes canales de longitud de onda.
El conjunto de bandas espectrales no está limitado estrictamente al espectro visible sino que también abarca el infrarrojo y el ultravioleta.

En la actualidad, el uso de imágenes hiperespectrales está aumentando considerablemente debido al lanzamiento de nuevos satélites y el interés en la observación remota de la Tierra,
que tiene utilidad en ámbitos tan diversos como defensa, agricultura de precisión, geología (detección de yacimientos minerales), valoración de impactos ambientales o incluso visión artificial.

\section{Objetivos}

El objetivo principal de este proyecto reside en dos puntos clave.
Por una parte la optimización del tiempo de inferencia en un modelo de machine learning usado para detectar desastres naturales mediante Openvino.
Finalmente el despliegue del modelo en un entorno cloud donde pueda funcionar como un servicio capaz de soportar miles de llamadas concurrentes.

\begin{itemize}
    \item Diseño de módulos individuales en VHDL que sirvan para realizar todas las operaciones que se necesitan para la implementación del algoritmo ATDCA-GS.
    \item Elaboración de una máquina de estados e implementacion del algoritmo usando los modulos individuales.
    \item Análisis y optimización de una implementación paralela previa en OpenCL del algoritmo.
    \item Obtención de resultados y realización de comparativas de rendimiento entre ambos lenguajes de programación.
\end{itemize}

%[CARLOS - INICIO]

%Hablar del objetivo general de realizar la implementación paralela en FPGA del algoritmo Automatic Target Detection and Classification Algorithm haciendo uso de la ortogonalización deGram Schmidt para el análisis de imágenes hiperespectrales.

%La consecución del objetivo general anteriormente mencionado se lleva a cabo en la presente memoria abordando una serie de objetivos específicos, los cuales se enumeran a continuación:

%\begin{itemize}
%\item
%\end{itemize}

%[CARLOS - FIN]


\section{Organización de esta memoria}

Teniendo presentes los anteriores objetivos concretos, se procede a describir la organización del resto de esta memoria, estructurada en una serie de capítulos cuyos contenidos se describen a continuación:

\begin{itemize}
    \item \textbf{Análisis hiperespectral}: se define el concepto de imagen hiperespectral y el modelo lineal de mezcla; se mencionan algunos sensores hiperespectrales (AVIRIS y EO-1 Hyperion) y algunas bibliotecas espectrales (USGS y ASTER); y por último, se presenta la necesidad de paralelización y las plataformas que se pueden utilizar para afrontar el problema de mejora de rendimiento.
    \item \textbf{Tecnología de las FPGAs}: se define de forma breve las tecnologías de las FPGAs.
    \item \textbf{Implementación}: se define el algoritmo ATDCA-GS en serie y se explica la paralelización y optimización que se ha llevado a cabo tanto en VHDL como en OpenCL.
    \item \textbf{Resultados}: se presentan los resultados obtenidos tras la implementación y ejecución del algoritmo en dispositivos FPGAs.
    \item \textbf{Conclusiones y trabajo futuro}: se presentan las principales conclusiones de los aspectos abordados en el trabajo a las que se han llegado y también algunas posibles líneas de trabajo futuro que se pueden desempeñar con relación al presente trabajo.
\end{itemize}

%[CARLOS - INICIO]

%Teniendo presentes los anteriores objetivos concretos, se procede a describir la organización del resto de esta memoria, estructurada en una serie de capítulos cuyos contenidos se describen a continuación:

%\begin{itemize}
%\item
%\end{itemize}

%[CARLOS - FIN]